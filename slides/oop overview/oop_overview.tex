\documentclass[aspectratio=169]{beamer}
% \usetheme{CambridgeUS}
% \usetheme{Boadilla}
\usepackage{
hyperref,
outlines,
amsmath,
amssymb,
amsthm,
enumitem,
listings,
graphicx}

% Code block environments
\definecolor{codegreen}{rgb}{0,0.6,0}
\definecolor{codegray}{rgb}{0.5,0.5,0.5}
\definecolor{codepurple}{rgb}{0.58,0,0.82}
\definecolor{backcolour}{rgb}{0.95,0.95,0.92}

\lstdefinestyle{mystyle}{
    backgroundcolor=\color{backcolour},   
    commentstyle=\color{codegreen},
    keywordstyle=\color{magenta},
    numberstyle=\tiny\color{codegray},
    stringstyle=\color{codepurple},
    basicstyle=\ttfamily\footnotesize,
    breakatwhitespace=false,         
    breaklines=true,                 
    captionpos=b,                    
    keepspaces=true,                 
    numbers=left,                    
    numbersep=5pt,                  
    showspaces=false,                
    showstringspaces=false,
    showtabs=false,                  
    tabsize=2,
    xleftmargin=.05\textwidth,
    xrightmargin=.05\textwidth
}
\lstset{language=Java,style=mystyle}
\setbeamertemplate{itemize item}[circle]

\title{Object Oriented Programming\\
\large Classes and Objects}
\author{Mark Kim}
\institute{San Francisco State University}
\date{3/9/2024}

\begin{document}

\frame{\titlepage}

\begin{frame}[fragile]
\frametitle{Classes}
Classes are an \alert{abstract} concept.  They are a blueprint for something
that belongs to that class.
\begin{example}
    In the following class, any person who is part of the Student class has
    an \verb|id| and a \verb|name|.
\begin{lstlisting}
class Student {
    String student_id;
    String name;
}
\end{lstlisting}
\end{example}
\end{frame}

\begin{frame}[fragile]
\frametitle{Objects}
Objects are the \alert{manifestation} of a Class.
\begin{example}
    \verb|student0| is a Student.  This Student therefore has a \verb|student_id| and a \verb|name|.
\end{example}
\vspace*{4mm}
This object can now be \alert{instantiated}.
\begin{example}
\begin{lstlisting}
Student student_0 = new Student();
\end{lstlisting}
\end{example}
\end{frame}

\begin{frame}[fragile]
\frametitle{Coding Classes and Objects}
Classes are created in their own \verb|.java| file.\\
These classes can then be used within other classes.

\begin{example}[student.java]
\begin{lstlisting}
class Student{
    public String id;
    public String name;

    public Student() { // notice that this is not static
        this.id = "000000000";
        this.name = "Default Student";
    }
}
\end{lstlisting}
\end{example}
\end{frame}

\begin{frame}[fragile]
\frametitle{Coding Classes and Objects, continued}
The class containing the \verb|main| method can then instantiate
objects from these classes.
\begin{example}[createStudent.java]
\begin{lstlisting}
class StudentDriver{
    public static void main(String[] args) {
        Student studentA = new Student();
    }

    System.out.println(studentA.id);
    // prints "000000000" to console
    System.out.println(studentA.name);
    // prints "Default Student" to console
}
\end{lstlisting}
\end{example}
\end{frame}

\begin{frame}[fragile]
\frametitle{Class attributes and behaviors}
Earlier, our student class only had attributes (and a constructor).\\
Classes also have behaviors.
\begin{example}[student.java]
\begin{lstlisting}
class Student{
    // attributes
    public String id;
    public String name;

    // behaviors; notice that these are all not static
    public Student() { // Constructor
        this.id = "000000000";
        this.name = "Default Student";
    }
    public compareName(Student b) {
        return this.name.compareTo(b.name);
    }
 }
\end{lstlisting}
\end{example}
\end{frame}

\end{document}